\documentclass[12pt]{article}

\usepackage{graphicx}
\usepackage{float}
\usepackage[latin1]{inputenc}
%% dieses package erlaubt, bei deutscher Tastatur Umlaute, ß direkt einzugeben

\usepackage{siunitx}

\textwidth=170mm
\textheight=240mm
\hoffset= -20mm       % may need change
\voffset= -25mm       % may need change



\begin{document}

%% we do the title page ourselves
\thispagestyle{empty}     % only for frontpage
\null
\vspace{40mm}
\begin{center}
{%%%%%%%%%%%%%%%%%%%%%%%%%% Titel
\Large  The Zeemann effect
\footnote{\noindent Experiment F44, performed on 3/20/18, Tutor Pooja, short evaluation}}\\[15mm]
%%%%%%%%%%%%%%%%%%%%%%%%%%% Authors
Lasse Gresista and Benjamin Haake

\vspace{25mm}

\parbox{0.9\textwidth}{   %% etwas schmaler als normaler Satz
Abstract:    
\small The abstract should preferentially be in English. Here we explain in a
few lines (i) what was done, and (ii) what the results were.
}
\end{center}

\vfill
%Attested as special evaluation: Date, Signature:
\vspace{20mm}

%% Rueckseite des Titelblatts leer. Bei einseitigem Druck entfernen
\newpage  
\null\thispagestyle{empty}

\newpage
\pagenumbering{arabic}

\section{Introduction}
\section{Setup of the experiment}
\section{Execution}
\section{Evaluation}
\section{Discussion}

\newpage
\begin{thebibliography}{00}   % {00}: max 2-stellige Referenznummer

\bibitem{sc} Zeemann-effect (2010)

\end{thebibliography}
\end{document}
