\documentclass[12pt]{article}

\usepackage{graphicx}
\usepackage{float}
\usepackage[latin1]{inputenc}
%% dieses package erlaubt, bei deutscher Tastatur Umlaute, ß direkt einzugeben

\usepackage{siunitx}

\textwidth=170mm
\textheight=240mm
\hoffset= -20mm       % may need change
\voffset= -25mm       % may need change



\begin{document}

%% we do the title page ourselves
\thispagestyle{empty}     % only for frontpage
\null
\vspace{40mm}
\begin{center}
{%%%%%%%%%%%%%%%%%%%%%%%%%% Titel
\Large  The Zeemann effect
\footnote{\noindent Experiment F44, performed on 3/20/18, Tutor Pooja, short evaluation}}\\[15mm]
%%%%%%%%%%%%%%%%%%%%%%%%%%% Authors
Lasse Gresista and Benjamin Haake

\vspace{25mm}

\parbox{0.9\textwidth}{   %% etwas schmaler als normaler Satz
Abstract:    
\small The abstract should preferentially be in English. Here we explain in a
few lines (i) what was done, and (ii) what the results were.
}
\end{center}

\vfill
%Attested as special evaluation: Date, Signature:
\vspace{20mm}

%% Rueckseite des Titelblatts leer. Bei einseitigem Druck entfernen
\newpage  
\null\thispagestyle{empty}

\newpage
\pagenumbering{arabic}

\section{Introduction}
In this experiment the effect of the normal Zeeman effect on the red spectral line of the Cadmium spectrum is observed using a Lummer-Gehrcke Plate. Additionally the wavelength of the spectral line is measured using a Czerny-Turner spectrometer.

\subsection{•}

\section{Setup of the experiment}
\subsection{Part1: Spectroscopy of the Zeeman Effect}
To investigate the normal Zeeman effect, the light of a Cadmium Lamp is observed through a Lummer-Gehrcke Plate. The pole pieces of an electromagnet are positioned on opposing sides of the CD lamp. The set-up is turnable and the pole pieces have a hole in the middle, which makes it possible to observe both transversal and longitudinal directions. The spectra are taken through a red light filter via a telescope and a CCD-Camera. The magnetic field strength is measured with a Gaussmeter.
\subsection{Part2: Precision Spectroscopy}
To determine the absolute wavelength of the red CD line a Czerny-Turner Spectrometer is used. Light entering the spectrometer through an entrance split is focused by a concave mirror on a grating, where it gets diffracted. The diffracted light then gets focused by another concave mirror onto a CCD camera. The light of the CD lamp and the neon lamp, which is used for calibration of the setup, is connected to the spectrometer using an optical fiber and two lenses to focus the light on the entrance slit. 

\section{Execution}
\subsection{Part1: Spectroscopy of the Zeeman Effect}
To convert the current applied to the electromagnets into the magnetic field strength, the field strength is measured for currents from 1-14A in 1A steps. This is done for ascending and descending currents to account for the hysteresis effect. Each measurement is done 3 times and an average is taken.
\\Next the light of the CD lamp with the magnetic field turned on is observed in both longitudinal and transversal directions. In the transversal direction spectra are recorded for a current of I=10A, I=12 and I=14A in ascending order. Additionally $\lambda/4$ and polarisation filters are used to determine the polarisation of the emitted light. 
\subsection{Part2: Precision Spectroscopy}
Spectra where taken with a grating of 1800 lines/mm and an integration time of 60 seconds. First the region of the neon spectrum with wavelengths of around 630-655nm was observed. Without changing the setup except now using the CD lamp the same region of the CD spectrum was observed. 
\section{Evaluation}
\subsection{Part1: Spectroscopy of the Zeeman Effect}

\subsection{Part2: Precision Spectroscopy}


\section{Discussion}

\newpage
\begin{thebibliography}{00}   % {00}: max 2-stellige Referenznummer

\bibitem{sc} Zeemann-effect (2010)

\end{thebibliography}
\end{document}
